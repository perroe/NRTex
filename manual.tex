% Specify language of report

\documentclass[screen,note,british,12pt]{nrdoc}


% -------------------------------------------------
% BEGIN nrdoc special commands
\ifx\nrdocument\undefined\relax\else
\reportnumber{sand/03/03}
\keywords{\LaTeX, NR}
\project{}
\projectnumber{}
\frontpagefigure{figs/texfriendly} % No file extension.
\target{All employees}
\researchfield{}

%\nrtitle{User manual for nrdoc: a \LaTeX\ class for notes and reports}
%\shorttitle{User manual for nrdoc}
\fi
% END nrdoc special commands
% ------------------------------------------------


\makeindex

\begin{document}



\title{User manual for nrdoc: a \LaTeX\ class for notes and reports}
\author{Harald Soleng\and Anders L�land}
\date{September 2003}
\maketitle

% The abstract must always be present
\begin{abstract}
This document gives an example of the use of the \texttt{nrdoc} class. The
purpose of this document class is to provide a unified style for NR
notes and reports conforming with standard typographical practice in
Norwegian and English.  
\end{abstract}

\tableofcontents

\mainmatter

\section{Introduction}
This document gives an example of the use of the \texttt{nrdoc} class. 
In Appendix \ref{sec:code} the \LaTeX\ source code for this document is
given.  

\section{Two ways to use the nrdoc class}

The \texttt{nrdoc} class can be used both by \LaTeX\ (using the
\texttt{latex} command) and pdf\LaTeX\ (using the \texttt{pdflatex}
command). Using the option 
\textit{screen} \index{screen}\index{option!screen}
together with \texttt{pdflatex} or \texttt{latex} and \texttt{dvipdfm}
gives an \textbf{electronic report with hyperlinks}.\index{hyperlinks}

\subsection{Including a figure}

On the front page, a figure\index{figure} 
can be included with \texttt{frontpagefigure}.
Just give the full path to the figure file, without the file extension.
The figure is automatically scaled to fit the front page.

Other
figures are included by \texttt{includegraphics}. An example is given
here:
\begin{verbatim}
\begin{figure}[ht]
  \begin{center}
    \includegraphics[width=8cm,angle=45]{figs/fotball}
    \caption{\label{fig:logo} Rolling NR logo.}
  \end{center}
\end{figure}
\end{verbatim}
\begin{figure}[ht]
  \begin{center}
    \includegraphics[width=8cm,angle=45]{figs/fotball}
    \caption{\label{fig:logo} Rolling NR logo.}
  \end{center}
\end{figure}
Note that the extension of the figure file name should be dropped. The
reason is that \LaTeX\ will find the required file itself. 
The figure
files should have the following format:\index{fornmat}
\begin{itemize}
\item \texttt{latex}: postscript [.ps] or encapsulated postscript [.eps]
\item \texttt{pdflatex}:pdf  [.pdf] or jpeg [.jpeg/.jpg]
\end{itemize}

\section{Fonts}

The default size is 11pt, but 10pt and 12pt sizes are also supported
through standard 
options.\index{option!10pt}\index{option!11pt}\index{option!12pt}

The standard font is type 1 versions of computer modern. The exception is
the typewriter font which has been replaced with almost European fonts.
There is also the option \textit{times} giving built-in times-roman fonts.
These fonts are used on the cover and the title page.

\section{Mathematics}\index{mathematics}

The \texttt{nrdoc} class loads the ams\index{amslatex@ams\LaTeX} 
packages so that the full
power of these extensions are available.
\begin{align}\label{eq:ey}
\begin{split}
  EY(s)&=\sum_{i=1}^n \mathcal{K}_s(s_i)Ew(s_i),\\
  \gamma(s,s')&=\sum_{i=1}^n \mathcal{K}_s(s_i)
  \mathcal{K}_{s'}(s_i)Ew^2(s_i).
\end{split}
\end{align}
Equations \eqref{eq:ey} are very nice equations.


\section{Citation using \textsc{Bib}\TeX}

Citations\index{cite} are handled by \textsc{Bib}\TeX\ and a style file
similar to \texttt{natbib}. This style file is automatically loaded by
\texttt{nrdoc}. Some examples
are given in Table \ref{tab:cite}.
\begin{table}[ht]
\centering
\begin{tabular}{lll}
\hline
usual citations &  
\protect\cite{host02} & 
\verb!\cite{host02}! \\
in parentheses  &  
\protect\citep{host02} & 
\verb!\citep{host02}!\\
with a see     &  
\protect\citep[see][]{host02} &
\verb!\citep[see][]{host02}!\\
with a section/page & 
\protect\citet[Section 2]{host02} & 
\verb!\citet{host02}!\\
will all authors & 
\protect\citet*{host02} & 
\verb!\citet*{host02}!\\
without parentheses & 
\protect\citealt{host02}&
\verb!\citealt{host02}!\\
only the year & 
\protect\citeyear{host02} & 
\verb!\citeyear{host02}!\\
only the author & 
\protect\citeauthor{host02} &
\verb!\citeauthor{host02}! \\
\hline
\end{tabular}
\caption{Examples of citations using \textsc{Bib}\TeX{}.}
\label{tab:cite} 
\end{table}
The \textsc{Bib}\TeX{} file (\texttt{ref.bib}) is given in 
Appendix \ref{sec:ref.bib}. 
A simpler numerical\index{numeric citation label} 
citation scheeme
with citations appearing in the order of citations is supported as well
using the option \textit{citenumeric}.

\section{Languages}

The following options are available for 
different language\index{language} support:
\begin{description}
\item[\textit{norsk}] bokm\aa l.\index{option!norsk}
\item[\textit{nynorsk}] nynorsk.\index{option!nynorsk}
\item[\textit{american}] u.s.\ english (default)\index{option!american}
\item[\textit{british}] u.k.\ english \index{option!british}
\end{description}

\section{The draft option}

The draft option prevents figure inclusion, turns off the cover generation 
and adds
marks on oversized lines.

\section{Making an index}

There is support for making indexes with makeindex.
Add the command \texttt{\\makeindex} in the preamble, and
run \texttt{makeindex} on the main tex file. This generates
an index file. It is called at \verb!\end{document}!.


\section{Long reports}

The class option \textit{long}\index{option!long} 
gives access to part\index{part} and chapter\index{chapter} commands.
By default the highest level is section.

\bibliography{ref}

\clearpage
\appendix

\section{The \LaTeX{} code for this document\label{sec:code}}
\verbatiminput{manual.tex}
\section{The \textsc{Bib}\TeX{} file (\texttt{ref.bib})}\label{sec:ref.bib}
\verbatiminput{ref.bib}

\end{document}






