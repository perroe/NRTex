% Specify language of report

\documentclass[note,screen,british,12pt]{nrdoc}

% -------------------------------------------------
% BEGIN nrdoc special commands
\ifx\nrdocument\undefined\relax\else
\reportnumber{sand/03/03}
\keywords{\LaTeX, NR}
\project{}
\frontpagefigure{figs/texfriendly} % No file extension.
\target{All employees}
\researchfield{}
\fi
% END nrdoc special commands
% ------------------------------------------------


\makeindex

\begin{document}



\title{User manual for nrdoc: a \LaTeX\ class for notes and reports}
\author{Harald Soleng\and Anders L�land}
\date{\today}
\maketitle

% The abstract must always be present
\begin{abstract}
This document gives an example of the use of the \texttt{nrdoc} class. The
purpose of this document class is to provide a unified style for NR
notes and reports conforming with standard typographical practice in
Norwegian and English.  
\end{abstract}

\tableofcontents

% The mainmatter must always be present
\mainmatter

\section{Introduction}
This document gives an example of the use of the \texttt{nrdoc} class. 
In Appendix \ref{sec:code} the \LaTeX\ source code for this document is
given.  

\section{Two ways to use the nrdoc class}

The \texttt{nrdoc} class can be used both by \LaTeX\ \index{latex@\LaTeX} 
\citep{latex}
(using the \texttt{latex} command) and pdf\LaTeX\ \index{pdflatex@pdf\LaTeX}
(using the \texttt{pdflatex} command). Using the option
\textit{screen} \index{screen}\index{option!screen} together with
\texttt{pdflatex} or \texttt{latex} and
\texttt{dvipdfm}\index{dvipdfm} gives an \textbf{electronic report
  with hyperlinks}.\index{hyperlinks} An example of the hyperlinks is
given here: If this is an electronic report, Equations \eqref{eq:ey}
can be found by clicking the equation number. The option
\textit{screen} gives the lines on the front page and hyperlinks, so
this option should not be used for a paper report. 

\subsection{Note and report}
The default option \textit{note}\index{option!note} gives a NR
note. The option  \textit{report}\index{option!report} gives a NR
report. For a NR report, these additional variables must be
specified:



\subsection{Long reports}

The class option \textit{long}\index{option!long} 
gives access to part\index{part} and chapter\index{chapter} commands.
By default the highest level is section\index{section}.

\subsection{Closed}
The option \textit{closed}
\index{option!closed} gives a closed report. 

\section{Front and title page}

For the title page to work, the abstract\index{abstract} (summary) must be
present. The title is set by \verb!\title!. \index{title}
\verb!\nrtitle! \index{nrtitle} can be used to specify the title in
the title page (this is convenient if the title on the front page
includes some line shifts). \verb!\shorttitle! \index{nrtitle}
specifies the text in the heading of each page. If \verb!\shorttitle!
\index{shorttitle} is not specified, the text in the heading will be
taken from \verb!\title!. 

The date\index{date} is taken from \verb!\date!.

The note or publication number is given by
\verb+\reportnumber+\index{reportnumber@\verb+\reportnumber+}.
The keywords of the document are given by
\verb+\keywords+\index{keywords@\verb+\keywords+}.
The project number is given by 
\verb+\project+\index{project@\verb+\project+}, and the target group by
\verb+\target+\index{target@\verb+\target+}.
The research field is specified using
\verb+\researchfield+\index{researchfield@\verb+\researchfield+}.

For reports, you also need to supply the isbn\index{isbn number} number. 
The isbn number consists
of a country code, a publisher code, 
a publication number and a control digit.
You can either use \texttt{isbn}\index{isbn@\verb+\isbn+} to supply 
the full number or 
\texttt{shortisbn}\index{shortisbn@\verb+\shortisbn+}
to supply just the last five digits:
\begin{verbatim}
\isbn{82-539-0481-9}
\shortisbn{0481-9}
\end{verbatim}

A typical preamble\index{preamble} then looks as follows:
\begin{verbatim}

\documentclass[note,screen,british,12pt]{nrdoc}
\reportnumber{sand/03/03}
\keywords{\LaTeX, NR}
\project{}
\frontpagefigure{figs/texfriendly} % No file extension.
\target{All employees}
\researchfield{}

\makeindex
\end{verbatim}
At the beginning of the document, we specify the title,
author(s) and abstract:
\begin{verbatim}
\begin{document}
\title{}
\author{}
\date{\today}
\maketitle

\begin{abstract}
The abstract is obligatory!
\end{abstract}

\tableofcontents 	% optional
\mainmatter 		% required

\end{verbatim}
Note that a \verb+\mainmatter+\index{mainmatter@\verb+{\mainmatter}+}
command is required before the first section or chapter.

\section{Including a figure}

On the front page, a figure\index{cover figure} can be included with
\texttt{frontpagefigure}. Just give the full path to the figure file,
without the file extension. The figure is automatically scaled to fit
the front page. 

Other figures\index{graphics}\index{illustration} \index{figure}
are included by \texttt{includegraphics}. An example is
given here (and in Figure \ref{fig:logo}):
\begin{verbatim}
\begin{figure}[ht]
  \begin{center}
    \includegraphics[width=8cm,angle=45]{figs/fotball}
    \caption{\label{fig:logo} Rolling NR logo.}
  \end{center}
\end{figure}
\end{verbatim}
\begin{figure}[ht]
  \begin{center}
    \includegraphics[width=8cm,angle=45]{figs/fotball}
    \caption{\label{fig:logo} Rolling NR logo.}
  \end{center}
\end{figure}
Note that the extension of the figure file name should be dropped. The
reason is that \LaTeX\ will find the required file itself. 
The figure
files should have the following format:\index{format}
\begin{description}
\item [\LaTeX] postscript [.ps] or encapsulated postscript
  [.eps] \index{format!postscript}
\item [pdf\LaTeX] pdf  [.pdf] or jpeg [.jpeg/.jpg]
  \index{format!pdf} \index{format!jpeg}
\end{description}

\subsection{Format conversion}
postscript files can be converted to pdf using the \texttt{ps2pdf}
command or Acrobat Distiller. 
pdf files can be converted to postscript using the \texttt{pdf2ps}
command or Adobe Acrobat. 

\section{Fonts}

The default size is 11pt, but 10pt and 12pt sizes are also supported
through standard 
options.\index{option!10pt}\index{option!11pt}\index{option!12pt}

The standard font is type 1 versions of computer modern. The exception is
the typewriter font which has been replaced with almost European fonts.
There is also the option \textit{times} \index{option!times} giving
built-in times-roman fonts.  These fonts are used on the cover and the
title page. 

\section{Mathematics}\index{mathematics}

The \texttt{nrdoc} class loads the ams\index{amslatex@ams\LaTeX} 
packages so that the full power of these extensions are available.
\begin{align}\label{eq:ey}
\begin{split}
  EY(s)&=\sum_{i=1}^n \mathcal{K}_s(s_i)Ew(s_i),\\
  \gamma(s,s')&=\sum_{i=1}^n \mathcal{K}_s(s_i)
  \mathcal{K}_{s'}(s_i)Ew^2(s_i).
\end{split}
\end{align}

\section{Citation using {Bib}\TeX}
 
Citations\index{cite}
are handled by {Bib}\TeX\ 
\index{bibtex@BiB\TeX}\index{cite!BiB\TeX@bibtex} and 
\texttt{natbib} \citep{natbib:7.1}. By default 
an author--year citation 
scheme\footnote{A simpler numerical\index{cite!numeric
citation label} citation scheme with citations appearing in the
order of citations is supported as well using the option
\textit{citenumeric}. This is intended for documents that need
numerical citations due to external demands.}  is used.
The required style files are 
automatically loaded by \texttt{nrdoc}. Some
examples are given in Table \ref{tab:cite}.    
\begin{table}[ht]
\centering
\begin{tabular}{lll}
\hline
usual citations &  
\protect\cite{mardia79} & 
\verb!\cite{mardia79}! \\
in parentheses  &  
\protect\citep{mardia79} & 
\verb!\citep{mardia79}!\\
with a see     &  
\protect\citep[see][]{mardia79} &
\verb!\citep[see][]{mardia79}!\\
with a section/page & 
\protect\citet[Section 2]{mardia79} & 
\verb!\citet{mardia79}!\\
will all authors & 
\protect\citet*{mardia79} & 
\verb!\citet*{mardia79}!\\
without parentheses & 
\protect\citealt{mardia79}&
\verb!\citealt{mardia79}!\\
only the year & 
\protect\citeyear{mardia79} & 
\verb!\citeyear{mardia79}!\\
only the author & 
\protect\citeauthor{mardia79} &
\verb!\citeauthor{mardia79}! \\
\hline
\end{tabular}
\caption{Examples of citations using \textsc{Bib}\TeX{}.}
\label{tab:cite} 
\end{table}
The citations are listed in a  \textsc{Bib}\TeX{} file with the extencions
\texttt{.bib}. An example (\texttt{ref.bib}) is given in Appendix
\ref{sec:ref.bib}. Use the \texttt{bibtex} command to run   
\textsc{Bib}\TeX{}.


\section{Languages}

The following options are available for 
different language\index{language} support:
\begin{description}
\item[\textit{american}] us english (default)\index{option!american} 
\item[\textit{british}] uk english \index{option!british} 
\item[\textit{english}] uk english \index{option!english} 
\item[\textit{norsk}] bokm\aa l.\index{option!norsk} 
\item[\textit{nynorsk}] nynorsk.\index{option!nynorsk} 
\item[\textit{UKenglish}] uk english \index{option!UKenglish}
\item[\textit{USenglish}] uk english \index{option!USenglish} 
\end{description}

\section{The draft option}\index{option!draft}

The draft option prevents figure inclusion, turns off the cover
generation and adds marks on oversized lines.

\section{Making an index}\index{index}

There is support for making indexes with makeindex.
Add the command \verb!\makeindex! in the preamble, and
run \texttt{makeindex} on the main tex file. This generates
an index file. It is called at \verb!\end{document}!.


\section{Default packages}
The following packages are loaded by default:
\begin{description}
\item[\textit{inputenc}] Handles Norwegian
  letters. \index{package!inputenc} 
\item[\textit{amsmath}] For advanced math. \index{package!amsmath} 
\item[\textit{paralist}] For list
  environments. \index{package!paralist} 
\item[\textit{fancyheadings}] For
  headers. \index{package!fancyheadings} 
\item[\textit{verbatim}] For verbatim text. \index{package!verbatim} 
Also contains a useful \texttt{comment} environment.
\item[\textit{xspace}] For space control. \index{package!xspace}
\item[\textit{rotating}] For text rotation. \index{package!rotating} 
\item[\textit{makeidx}] For index. \index{package!makeidx}
\item[\textit{multicol}] For multiple columns. \index{package!multicol} 
\item[\textit{color}] For colour. \index{package!color}
\item[\textit{graphicx}] for graphics
  inclusion. \index{package!graphicx} 
\item[\textit{babel}] For language support. \index{package!babel}
\item[\textit{geometry}] For page geometry. \index{package!geometry} 
\item[\textit{hyperref}] For hyperlinks. \index{package!hyperref}
\end{description} 
For more information about most of these packages,
consult the book by \cite{latexcompanion}.

\bibliography{ref}

\clearpage
\appendix

\section{Setting up Unix variables}
For the system to find the \texttt{nrdoc} class, the variable
\texttt{TEXINPUTS}\index{texinputs@TEXINPUTS} needs to be set. 
If you use \texttt{tcsh} (that is
you have a \texttt{.cshrc} file), the  \texttt{.cshrc} file should
include this line:
\begin{verbatim}
setenv TEXINPUTS .:/nr/group/maler/nrdoc:
\end{verbatim}

\section{The \LaTeX{} code for this document\label{sec:code}}
\verbatiminput{manual.tex}
\section{The {Bib}\TeX{} file (ref.bib)}\label{sec:ref.bib}
\verbatiminput{ref.bib}

\end{document}






